\chapter*{Abkürzungsverzeichnis}
\addcontentsline{toc}{chapter}{Abkürzungsverzeichnis} % Hinzufügen zum Inhaltsverzeichnis 

\begin{acronym}[WYSISWG] % längstes Kürzel wird verw. für den Abstand zw. Kürzel u. Text

	% Alphabetisch selbst sortieren - nicht verwendete Kürzel rausnehmen!

	\acro{AOP}{Aspect-oriented Programming}
	\acro{API}{Application Programming Interface}
	\acro{CD}{Continuous Delivery}
	\acro{CI}{Continuous Integration}
	\acro{DSL}{Domain Specific Language}
	\acro{DTO}{Data Transfer Objects}
	\acro{HTTP}{Hypertext Transfer Protocol}
	\acro{IDE}{Integrated Development Environment}
	\acro{IO}{Input/Output}
	\acro{IoC}{Inversion of Control}
	\acro{JPA}{Java Persistence API}
	\acro{JSON}{JavaScript Object Notation}
	\acro{JTA}{Java Transaction API}
	\acro{JVM}{Java Virtual Machine}
	\acro{MVC}{Model View Controller}
	\acro{OAI}{OpenAPI}
	\acro{ORM}{Object-relational Mapping}
	\acro{POI}{Point of interest}
	\acro{POJO}{Plain Old Java Object}
	\acro{QoS}{Quality of Service}
	\acro{REST}{Representational State Transfer}
	\acro{RPC}{Remote Procedure Call}
	\acro{SDK}{Software Development Kit}
	\acro{SOAP}{Simple Objec Access Protocol}
	\acro{SQL}{Structured Query Language}
	\acro{UI}{User Interface}
	\acro{URI}{Uniform Resource Identifier}
	\acro{WWW}{World Wide Web}
	\acro{XML}{Extensible Markup Language}
	\acro{YAML}{YAML Ain’t Markup Language}

\end{acronym}