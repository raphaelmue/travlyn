\chapter{Ausblick}

	Um die zugrunde liegende Vision dieses Projekts zu erfüllen, könnten weitere Funktionalitäten implementiert werden, die nicht Teil des Scopes in dieser Entwicklung waren. Im Folgenden wird auf diese Möglichkeiten eingegangen und damit aufgezeigt, was in zukünfigen Projekten und Erweiterungen der \textit{Travlyn} Applikation umgesetzt werden könnte.
	
	\section{Namenszuordnung der Stops}
	
		Um eine vollständige Liste von Stops mit zusätzlichen Informationen innerhalb einer Stadt zu erhalten, wird eine vollständige Namenszuordnung für alle Städte benötigt (\autoref{implementation.mapping}). Diese Namenszuordnung kann nun entweder über einen externen Dienstleister oder aber eigens erfolgen. 
		
		Wichtig ist hierbei auch, dass eine Verifikation -- eventuell auch durch den Nutzer -- erfolgen kann. Eine solche Verifikation könnte so aussehen, dass der Nutzer falsch zugeordnete Informationen -- beispielsweise wird statt der Beschreibung der Löwenstatue auf dem Trafalgar Square eine Beschreibung des Tiers angezeigt -- melden kann, sodass diese auf der Datenbank oder beim Mapping korrigiert werden können. 
		
	\section{Markieren von Stops}
	
		Da \textit{Travlyn} aktuell sehr von den Daten von OpenRoute Service und DBpedia abhängig ist, wäre eine Erweiterung der Funktionalität das Markieren von Stops: Wenn ein Nutzer an einer Sehenswürdigkeit vorbei kommt, kann er diese Lokation als Sehenswürdigkeit markieren und entsprechende Informationen -- wie Namen, Beschreibung, Preis etc. -- hinzufügen. Für eine Verifikation dieser Sehenswürdigkeit kann aus zwei Ansätzen gewählt werden: 
		
		\begin{itemize}
			\item \textbf{Interne Verifikation}: Eine Möglichkeit der Verifikation ist, dass ein Teil des \textit{Travlyn} Teams solche Markierungen überprüft. Dieser Ansatz ist jedoch zeitaufwändig und erhöht zudem die Kosten für die Pflege der Applikation. 
			\item \textbf{Crowd Sourcing}: Bei einem Crowd Sourcing basiertem Ansatz werden die Daten durch die Nutzer und das System verifiziert. Dies erfordert jedoch eine Art der Spam-Identifikation, um ungültige Stops auszuschließen. Wenn ein Nutzer mehrfach durch Spam auffällt, soll der Nutzer gesperrt werden. Dies ist Aufgabe von \textit{Travlyn} und erfordert zusätzlichen Implementationsaufwand. Projekte, die einen solchen Crowd Sourcing basiertem Ansatz verwenden, wie Wikipedia, können als Vorbild für die Implementation fungieren. 
		\end{itemize}
	
		Vorstellbar wäre außerdem auch ein Belohnungssystem (eine Art Gamification \cite{Zichermann.2011}), um Nutzer aufzufordern, die Datenbasis von \textit{Travlyn} zu pflegen und zu erweitern. 
				
	\section{Automatische Generierung von Trips}
	
		Ein weiteres Feature könnte die automatische Generierung von Trips sein. Ausgehend von den Präferenzen des Benutzers werden entsprechende Sehenswürdigkeiten in einer Stadt gefiltert und als Trip abgespeichert. Diese Optimierung berücksichtigt auch Parameter wie die verfügbare Zeit oder das Budget des Nutzers. Wenn beispielsweise ein Nutzer sehr an kulturellen Einrichtungen interessiert ist, jedoch nicht so sehr an Technik, bieten sich hierfür \zB Kunstmuseen oder -galerien an. Es ist allerdings eine gepflegte Datenbasis erforderlich, um eine solche Generierung zu ermöglichen. 
		
		Auch könnten abhängig von den Präferenzen den Nutzers mittags oder abends ein Restaurant oder Imbiss zum Trip hinzugefügt werden, oder aber auch kulturelle Abendveranstaltungen und regionale Events. Das setzt aber eine Datenquelle voraus, die genau solche Informationen bzw. Daten liefert. Außerdem wäre eine Anpassung des Datenschemas notwendig, um diese zusätzlichen Informationen zu persistieren. 
		
	\section{Audioguides \& Reisegruppen}
	
		Um nun das Reiseerlebnis zu einer privaten und selbst gestaltbaren Führung durch die Stadt zu verwandeln, könnte mittels Text-To-Speech ein Trip mit einer zusätzlichen Audiospur unterstützt werden. So liest \textit{Travlyn} nicht nur die Navigationsanweisungen vor, sondern ebenso die zusätzlichen Informationen zu dem aktuellen Stop, bei welchem sich der Nutzer befindet. 
		
		Vervollständigt würde dies durch die Möglichkeit, einen Trip in einer Gruppe von Nutzern durchzuführen. Alle Nutzer in der Gruppe würden denselben Trip gleichzeitig ausführen und dabei auch gleichzeitig die gleichen Informationen erhalten. 
		
	\section{Veröffentlichung}
	
		Um Travlyn zu veröffentlichen und eventuell auch zu kommerzialisieren müssten folgende Überlegungen ausgearbeitet und umgesetzt werden:
		
		\begin{itemize}
			\item \textbf{Serverlandschaft}: 
				Es müsste zunächst sichergestellt sein, dass \textit{Travlyn} auch eine hohe Verfügbarkeit und Performance aufweist. Der Aufbau einer Serverarchitektur bzw. -landschaft ist ein Schritt, um diese beiden nicht-funktionalen Ziele zu gewährleisten. 
			\item \textbf{Datenbasis}: 
				Außerdem muss die Datenbasis von \textit{Travlyn} ausreichend sein, sodass es dem Nutzer möglich ist, für die meisten Städte dieser Welt, Trips zu generieren. Dies beinhaltet vor allem die bereits beschriebene Namenszuordnung aber eventuell auch das Markieren von Stops. \\
				Wie bereits in \autoref{implementation.apilimits} erwähnt, könnte der Einsatz eines initialen Skriptes dabei helfen, die Datenbasis aufzubauen und eine Grundlage zu schaffen. Sobald das erfolgt ist, werden die restlichen Daten durch Anfragen der Benutzer in die Datenbank von \textit{Travlyn} nachgeladen. \\
				Es müsste außerdem gewährleistet sein, dass die Daten aktuell bleiben und nicht auf der Datenbasis veralten. Die Aktualisierung könnte entweder über einen Hintergrundprozess laufen, der nach und nach alle Datensätze wiederholt aktualisiert oder über eine Event-basierte Aktualisierung der Daten (durch den \acs{API} Anbieter oder durch den Nutzer). 
			\item \textbf{Externe \acs{API}s}: 
				Bei einer hohen Zahl an Nutzern muss darüber hinaus gewährleistet sein, dass \textit{Travlyn} ein ausreichendes Quota bei externen \acs{API}s, wie OpenRoute Service und DBpedia, hat. 
			\item \textbf{Veröffentlichung der Android App}: 
				Um eine möglichst große Menge an Kunden zu erreichen, bietet es sich an, die Android App im Google Play Store zu veröffentlichen. Das erzeugt ein Vertrauen zum Kunden, da Apps im Google Play Store auch durch Google verifiziert werden. Dafür müssten jedoch noch ein paar rechtliche Anpassungen erfolgen. 
			\item \textbf{Marketingkonzept}: 
				Damit \textit{Travlyn} auch Profit erwirtschaftet, muss zum einen ein Preismodell erarbeitet werden -- entweder auf Lösungs- oder Lizenzbasis -- und außerdem ein Marketingkonzept, welches das Schalten von Werbung oder aber auch Marketingkampagnen beinhaltet.\\
				\textit{Travlyn} könnte aber auch als Werbeplattform für die entsprechenden Sehenswürdigkeiten fungieren, indem diese Sehenswürdigkeiten explizit angezeigt -- beispielsweise als eine markierte Anzeige oder ganz oben in der Liste der Stops -- oder in die automatisch generierten Trips eingebaut werden. Somit könnte \textit{Travlyn} auf der einen Seite Umsatz erzielen und auf der anderen Seite die Betreiber der Sehenswürdigkeiten unterstützen. 
		\end{itemize}
	
		
	