\chapter{Implementierung}

	In diesem Kapitel werden einige Implementationsdetails dargestellt und erläutert, welche entweder einen kritischen Aspekt des Projekts darstellen oder von hoher Relevanz für den Erfolg des Projektes sind. 
	
	\section{Controller Interfaces}
	
		Wie bereits erwähnt sind die sog. Controller in einer \acs{MVC} Architektur für den Behandlung von Benutzereingaben zuständig. Im Falle einer \acs{REST}ful \acs{API} -- hier durch Spring umgesetzt -- dienen Controller dem Zweck, alle eingehenden \acs{HTTP} Anfragen zu behandeln. Dafür bietet Spring Annotationen, mit dessen Hilfe eindeutig definiert werden kann, wie die Anfrage auszusehen hat. Diese Definition eines sog. Endpoints beinhaltet immer einen Pfad, unter welchem der Endpoint erreicht werden kann, und eine \acs{HTTP} Methode. Es kann zusätzlich noch festgelegt werden, welche \acs{HTTP} Header bei einer Anfrage angegeben werden sollen und welcher \acs{MIME} Type zurück gegeben wird.
		
		\autoref{code:CityApi} zeigt beispielhaft, wie eine solche Definition eines Endpoints aussehen kann. Die eigentliche Beschreibung des Endpoints wird durch die Annotation \lstinline|@GetMapping| beschrieben. 	
		
		Um jedoch eine ausführlichere Beschreibung des Endpoints zu erstellen, können die von Swagger bereitgestellten Annotation verwendet werden. So lassen sich beispielsweise die Parameter, die Rückgabewerte aber auch die Authentifizierungsmethode beschreiben. Außerdem lassen sich noch Beschreibungen einfügen, um die Benutzung der \acs{API} zu erleichtern und zu dokumentieren. Für die City \acs{API} wird ein String als Parameter benötigt, mit dessen Hilfe die entsprechende Stadt zurückgegebene werden kann. Außerdem wird ein \acs{API} Key benötigt, um sich zu authentifizieren. 
		
		\lstinputlisting[
			label=code:CityApi,
			caption=Nutzung der beschriebenen Annotationen eines Controllers am Beispiel der City API,
			captionpos=b,               % Position, an der die Caption angezeigt wird t(op) oder b(ottom)
			style=EigenerJavaStyle,     % Eigener Style der vor dem Dokument festgelegt wurde
		]{code/CityApi.java}
		
	\section{Authentifizierung}
	
		\begin{figure}[ht!]
			\centering
			\includegraphics[width=1\textwidth]{images/authorization-flow-chart.pdf}
			\caption{Authentifizierungsprozess von \textit{Travlyn} mittels Spring Security Filter Chain}
			\label{fig:authenticationProcess}
		\end{figure} 
		
		\textit{Travlyn} stellt insgesamt zwei verschiedene Authentifizierungsrollen bereit. Wie \autoref{fig:UCD} zeigt, gibt es zum einen die Rolle des \acs{API}-Benutzers und die des registrierten Benutzers. Dem API Benutzer ist es erlaubt, Informationen wie beispielsweise über Trips anzufragen, wohingegen ein registrierter Benutzer auch die Möglichkeit hat, einen Trip zu erstellen. 
		
		Der Prozess der Authentifizierung basiert bei \textit{Travlyn} auf sog. Tokens. Sobald sich ein Benutzer registriert oder anmeldet, wird ein Token generiert -- das Token besteht aus 32 Zeichen und Ziffern und besitzt zudem ein Ablaufdatum, welches bei Überschreiten das Token als ungültig gekennzeichnet --, auf der Datenbank abgespeichert und zurück an den Benutzer gesendet. Um nun Endpoints anzufragen, welche eine Authentifizierung benötigen, wird dieses Token im \lstinline|Authorization|-Header im Format \lstinline|Bearer <token string>| gesendet. Auf dem Server erfolgt nun die Validierung des Tokens. So kann außerdem festgestellt werden, wer eine bestimmte Aktion ausgeführt hat. 
		
		Spring stellt für diesen Prozess einen Mechanismus bereit, um diese Authentifizierung möglichst einfach umzusetzen: Bevor eine Anfrage an einen bestimmten Endpoint zugelassen und die entsprechende Controller-Methode (wie bei \autoref{code:CityApi} beispielhaft gezeigt) aufgerufen wird, durchläuft diese Anfrage eine Reihe von Filtern. Diese sog. Filter Chain beinhaltet unter Anderem auch Authentifizierungsmechanismen, die sich eigens definieren lassen. So wurde für \textit{Travlyn} ein Filter namens \lstinline|AuthenticationTokenFilter| entwickelt, der dazu dient, das Token aus dem \acs{HTTP}-Header zu extrahieren und zu validieren. 
		
		Wenn das Token valide ist und auch der korrekten Benutzerrolle -- \acs{API}- oder registrierter Nutzer -- zugeordnet werden kann, wird die Anfrage an die entsprechende Controller Methode weitergeleitet und der entsprechende Nutzer im sog. \lstinline|SecurityContext| gespeichert, sodass auf die Rolle des Benutzers und auch auf den Benutzer selbst beim Prozessieren der Anfrage zugegriffen werden kann. Ist dies nicht der Fall, so wird diese Anfrage als nicht-authentifiziert markiert. \autoref{fig:authenticationProcess} visualisiert den Authentifizierungsprozess, wie er bei \textit{Travlyn} umgesetzt wurde.
		
		Um die zugelassenen Benutzerrollen für einen Endpoint der \acs{API} festzulegen, stellt das Modul Spring Security unter anderem die Annotation \lstinline|@PreAuthorize| zur Verfügung. Als Parameter wird ein in der \ac{SpEL} geschriebener Ausdruck übergeben, wie beispielsweise \lstinline|hasRole(API_USER)| in \autoref{code:CityApi}. So lässt sich die gesendete Authentisierung des Benutzers automatisiert überprüfen.
		
	\section{Trip Execution} 
	
		Einen großen Teil der Applikation stellt die Ausführung der Trips dar. Diese beinhaltet die Navigation des Benutzers durch die Stadt entlang der einzelnen, zu diesem Trip gehörigen Stops. Wenn der Nutzer einen Stop erreicht hat, werden zusätzliche Informationen zu diesem Stop angezeigt. Sobald die Sehenswürdigkeit nach dem Empfinden des Nutzers ausreichend besichtigt wurde, kann der Nutzer zum nächsten Stop navigieren. 
		
		Beim Starten dieser Navigation können zwei zusätzliche Parameter festgelegt werden: \lstinline|roundTrip| und \lstinline|reorderAllowed|. \textit{Travlyn} führt den Nutzer an seine Startposition zurück, wenn der erste Parameter auf \lstinline|true| gesetzt ist. Der zweite Parameter hingegen entscheidet, ob die Stops reorganisiert werden, sodass ein kürzerer Laufweg für den Nutzer entsteht. 
		
		Diese Reorganisierung der Stops entstammt aus der Menge der kombinatorischen Optimierungsproblemen und wird auch als \ac{TSP} bezeichnet. Dabei gilt es die kürzeste Strecke durch Rekombination der einzelnen Orte zu finden. Da dem Handlungsreisenden -- bzw. in diesem Fall dem Benutzer -- in jedem Schritt die Stops zur Auswahl stehen, welche noch nicht besucht wurden, existieren $(n - 1)!$ mögliche Touren, wobei $n$ die Anzahl der Orte -- bzw. Stops -- beschreibt. \acs{TSP} ist ein NP-schweres Problem, für welches unter der bisher unbewiesenen Annahme, dass die Komplexitätsklassen P und NP verschieden sind, kein Algorithmus existiert, der das Problem der kürzesten Rundreise in polynomieller Laufzeit lösen kann. \cite{Applegate.2006}
		
		\subsection*{Simulated Annealing}
		
		Simulated Annealing \cite{S.Kirkpatrick.1983} bezeichnet ein heuristisches Approximationsverfahren, welches das Ziel hat, eine Näherungslösung für Optimierungsprobleme wie \acs{TSP} zu finden. Der Algorithmus von Simulated Annealing ist durch physikalische Überlegungen motiviert: Der langsame Abkühlungsprozess von Metallen nach dem Erhitzen sorgt dafür, dass die Atome ausreichend Zeit haben, sich zu ordnen und stabile Kristalle zu bilden. Der Zustand, der auf diese Weise erzeugt wird, liegt somit nahe am Optimum und ist energiearm. 
		
		Wenn man das auf Optimierungsprobleme überträgt, entspricht die Temperatur $T$ der Wahrscheinlichkeit mit der sich ein Zwischenergebnis verschlechtern darf. So kann ein lokales Optimum wieder verlassen werden, was bei der lokalen Suche beispielsweise nicht möglich ist. Es werden also Verschlechterungen akzeptiert, da auf diese Weise die Wahrscheinlichkeit erhöht wird, ein besseres lokales Optimum und im besten Fall das globale Optimum zu erreichen. 
		
		Um dieses Verfahren durchzuführen, muss eine Metrik definiert werden, mit der sich die Qualität einer Lösung beschreiben lässt. Diese sog. Energie $E$ beschreibt im Falle von \acs{TSP} die Summe der Distanzen zwischen den einzelnen Stops $s = (x, y)$ einer Lösung $x$, wobei $n$ der Anzahl der Stops und $s_{start}$ dem Startpunkt entspricht:
		
		\begin{equation}
			E(x) = ||\left(s_{start}, s_1 \right)|| + \sum_{i = 1}^{n - 1} ||\left(s_i, s_{i + 1}\right)|| + ||\left(s_n, s_{start} \right)||
		\end{equation}
			
		Um nun festzustellen, ob eine Lösung $x_{2}$ besser oder schlechter als eine andere Lösung $x_{1}$ ist, lässt sich $\varDelta E$ bestimmen mit $\varDelta E = \left\| E\left(x_{2}\right) - E\left(x_{1}\right) \right\|$. 
		
		\begin{algorithm}
			\caption{Simulated Annealing Algorithmus \cite{S.Kirkpatrick.1983}}
			\label{alg:simulatedAnnealing}
			\begin{algorithmic}
				\State $T\gets T_{max}$
				\State $best\gets$ \textbf{\Call{\color{blue}init}{$ $}}
				
				\While{$T>T_{min}$}
				\State $next\gets $ \textbf{\Call{\color{blue}next}{$T, best$}}
				\State $\Delta E\gets$ \textbf{\Call{\color{blue}energy}{$next$}} $-$ \textbf{\Call{\color{blue}energy}{$best$}}
				\If{$\Delta E < 0$}
				\State $best\gets next$
				\ElsIf{\textbf{\Call{\color{blue}probability}{$T,\Delta E$}} $>$ {\Call{random}{$ $}}}
				\State $best\gets next$
				\EndIf
				\State $T\gets$ \textbf{\Call{\color{blue}cooling}{$T,best$}}
				\EndWhile \\
				\Return $best$
			\end{algorithmic}
		\end{algorithm}	
		
		
		\autoref{alg:simulatedAnnealing} beschreibt in Pseudocode die Vorgehensweise von Simulated Annealing. In jedem Iterationsschritt wird zunächst eine alternative Lösung und abhängig davon $\varDelta E$ bestimmt. Ist diese Lösung besser als die bisher beste, so wird sie übernommen. Falls nicht, wird mit einer bestimmten Wahrscheinlichkeit trotzdem zugelassen, dass die Lösung übernommen wird. Somit kann erreicht werden, dass aufgrund einer Verschlechterung ein besseres lokales Optimum gefunden wird. Diese Wahrscheinlichkeit ist definiert als
		
		\begin{equation}
			P \left(x\right) = e^{\frac{\varDelta E}{T}}
		\end{equation}
		
		wobei $T$ die Temperatur definiert. So ist bei hoher Temperatur die Wahrscheinlichkeit sehr hoch, dass eine schlechtere Lösung übernommen wird. Die Wahrscheinlichkeit sinkt jedoch mit fallender Temperatur. Wie bei dem abkühlenden Metall, kühlt die Temperatur mit jedem Iterationsschritt ab. Für diesen Abkühlungsprozess wird häufig eine Exponentialfunktion der Art 
		
		\begin{equation}
			f(T) = T_{max} T^{-\alpha}, 0.8 < \alpha < 1
		\end{equation}
				
		verwendet. Der Algorithmus terminiert, sobald die Temperatur $T$ unter einen Schwellenwert $T_{min}$ fällt. 
	
		Um die Distanz zwischen zwei Stops zu berechnen, wird als Näherung die Länge der Luftlinie berechnet: 
		
		\begin{equation}
			||(x_1, x_2)|| = \sqrt{ ( x_{2x} - x_{1x} )^2  + ( x_{2y} - x_{1y} )^2 }
		\end{equation}
		
		Dies entspricht nicht der korrekten Distanz, da diese sich an Straßen und Wegen in einer Stadt orientiert, jedoch müsste andernfalls für jede Distanzberechnung eine \acs{HTTP} Anfrage an Openroute Service (siehe \autoref{sec:openRouteService}) geschickt werden. Dies würde dazuführen, dass zum einen bei einer großen Anzahl an Stops das API Limit erreicht werden könnte und somit keine Anfragen mehr zugelassen werden, und zum anderen würde die Laufzeit für die Reorganisierung deutlich steigen.
		 