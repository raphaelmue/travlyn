\chapter{Einleitung}

	(Joshua)

	In der heutige Zeit unterliegt die Welt der Medien einer sehr rasanten und starken Veränderung. Es werden immer mehr neuartige und innovative Techniken entwickelt, wie Medien konsumiert werden können, z.B. Augmented und Virtual Reality. Durch diese Techniken wird versucht die Mediennutzung effektiver, intensiver und moderner zu machen. In vielen Bereichen haben diese Techniken bereits Einzug gehalten und sind für eine große Masse von Konsumenten verfügbar, z.B. Virtual Reality gaming mit Hilfsmitteln wie Oculus Rift \cite{OculusVR.3222020} und anderen Produkten.

	\vspace{0.25cm}

	Allerdings gibt es ebenfalls Bereiche bei denen die Digitalisierung und Nutzung neuer Techniken nicht komplett ausgenutzt werden und viele weitere Vorteile ungenutzt bleiben, wie z.B. beim Reisen \cite{Dredge.}: Viele Menschen nutzen Reiseführer, um sich einen Überblick über ihre Reisedestination zu verschaffen und einen grundlegenden Plan zu erstellen, allerdings findet man diese Reiseführer fast ausschließlich in gedruckter Buchform. Das bedeutet für die Buchform, dass immer schwere Printmedien mit in den Urlaub genommen werden müssen, dass Informationen in den Reiseführern nicht aktualisiert werden können, ohne eine neue Auflage herauszugeben und eine interaktive Gestaltung der Medien nicht möglich ist. Außerdem sind die Seiten in einem Buch begrenzt, d.h. es muss für jede Reise ein neuer Reiseführer erworben werden, der spezielle Informationen zum Reiseziel enthält. All dies sind Nachteile, die durch eine Digitalisierung der Inhalte ausgeglichen werden könnten: Das Smartphone ist heutzutage ein ständiger Begleiter, der ausgenutzt werden kann, um Echtzeitinformationen schnell und immer aktuell zur Verfügung zu stellen. Außerdem können interaktive Elemente wie Karten, Navigation, Audioguides uvm. direkt integriert werden. Es wäre möglich einen Anwendung zu schaffen, die in der Lage ist, für jede beliebige Stadt und Region auf der Welt Informationen bereit zu stellen, ohne dass neue Inhalte erworben werden müssen. Als Geschäftsmodel des Anbieters wäre es denkbar Geld durch den Einsatz von Werbung zu verdienen oder durch Bezahlung von Städten im Gegenzug für besondere Herausstellung innerhalb der Anwendung.  Damit könnte eine Reise entstehen, die unkomplizierter und trotzdem viel aktueller ist als bei Benutzung eines herkömmlichen Reiseführers.

	\vspace{0.25cm}

	Es könnten viele Services, die aktuell parallel zum Reiseführer genutzt werden (z.B. verschiedene Bewertungsportale und Karten) direkt integriert werden, um alle Informationen auf einen Blick zur Verfügung zu stellen. Ebenso könnte eine Personalisierung der verfügbaren Daten umgesetzt werden. Im Gegensatz zum herkömmlichen Reiseführer, welcher allgemeine und damit u.U. viele für den einzelnen irrelevant Informationen enthält, werden Vorschläge anhand der vom Nutzer gesetzten Vorlieben gemacht und somit nur nützliche Informationen zur Verfügung gestellt.

	\vspace{0.25cm}

	Insgesamt soll ein digitaler Reiseführer entstehen, der das Reiseerlebnis auf eine bessere, digitalere und einfachere Ebene hebt und noch mehr Spaß am Reisen erzeugen kann.

	\section{Motivation}

	Wir möchten mit dieser Arbeit einen Beitrag zu einer digitalisierten und technisch geprägten Gesellschaft leisten, indem wir eine Anwendung schaffen, die mit den Nachteilen von gedruckten Reiseführern aufräumt und eine neue innovative Art des Reisen schafft. Damit soll das Reiseerlebnis der Menschen verbessert werden und ihnen die Möglichkeit geben sich noch mehr auf das Erlebte zu konzentrieren, ohne Gedanken an eine komplizierte Planung, bei der viele Medien parallel genutzt werden, zu verschwenden.
	Außerdem ist diese Arbeit Teil unserer Prüfung zum Bachelor of Science und dient zur praktischen Anwendung der bisher im Studium erlernten Fähigkeiten und zum Ausprobieren neuer innovativer Techniken um unser Wissen zu erweitern.   

	\section{Gliederung der Arbeit}